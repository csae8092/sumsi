\input{latex-korrekturansicht-vorspann}
               \section[Hugo von Hofmannsthal an Arthur Schnitzler, 2. 10. {[}1898{]}]{ Hugo von Hofmannsthal an Arthur Schnitzler, 2. 10. {[}1898{]}}\nopagebreak\mylabel{v}\rehead{ }\normalsize\beginnumbering\briefempfaengerindex{Schnitzler, Arthur@\textsc{Schnitzler, Arthur}!zzzHofmannsthal, Hugo von@\emph{von Hugo von Hofmannsthal}!1898-10-021@{2. 10. {[}1898{]}}|(be} \toendnotes[C]{\smallbreak\pagebreak[2]} \Standort{CUL, Schnitzler, B 43.}
\physDesc{Brief, 1 Blatt, 3 Seiten
\newline{}Handschrift: schwarze Tinte, deutsche Kurrent
\newline{}Schnitzler: mit Bleistift die Jahreszahl ergänzt: »98« \newline{}Ordnung: 1) mit Bleistift von unbekannter Hand nummeriert: »\strikeout{127}« 2) mit Bleistift von unbekannter Hand nummeriert: »124«}\buchAbdrucke{\weitereDrucke{Hugo von Hofmannsthal, Arthur Schnitzler: \emph{Briefwechsel}. Hg. Therese Nickl und Heinrich Schnitzler. Frankfurt am Main: \emph{S. Fischer} 1964, S. 112.} }\toendnotes[C]{\smallbreak}\pstart
           \noindent{}\centering{}{\pb}\textcolor{gray}{\textbf{\textcolor{pink}{Hôtel de l’Europe Venise}}}\pend
           \pstart
           \noindent{}\centering{}\hspace*{5em}\textcolor{gray}{\textbf{sur le \textcolor{pink}{Grand Canal}}}\pend
           \pstart
           \noindent{}\centering{}\textcolor{gray}{\textbf{\textcolor{blue}{Marseille Frères}, Prop\textsuperscript{res}}}\pend
           \pstart
           \noindent{}\centering{}\textcolor{gray}{\textbf{Vue prise de l’hôtel}}\pend
           \pstart
           \raggedleft{}\textcolor{pink}{Venedig}{ }2\textsuperscript{ten} X.\pend
           \pstart{}mein lieber Arthur\pend\pstart
           ſo hör ich auf einmal von meinen \textcolor{blue}{Eltern}, daſs die Aufführung vom »\textcolor{green}{Vermächtnis}« unmittelbar bevorſteht und denke Sie auf den
                    Proben, in dem halbfinſteren Theater, u der Luft die Sie ſo gern haben und die
                    ich auch ſehr gern zu haben anfange. Dann kommen mir \textcolor{pink}{Wien}er Sommerabende ins Gedächtnis, das Bad im \textcolor{pink}{Neufchatelerſee}, der letzte {\pb}Tag am Dampfſchiff und ich
                    denke mir, wie ſchön und gut es iſt, was für ein großes Glück, daſs ich Menſchen
                    wie Sie ſo früh hab finden und behalten dürfen.\pend
           \pstart
           Ich war bei den \textcolor{pink}{Thürmen}, von
                    denen Sie mir einen geſchenkt\strikeout{en} haben, dann in
                        \textcolor{pink}{Florenz}, worüber mehr als viel zu erzählen
                    iſt und ſitze nun ſeit 14 Tagen hier ſo fieberhaft fleißig wie ichs manchmal und
                    leider ſo ſelten ſein kann. Etwa den 10\textsuperscript{ten} bin ich
                    in \textcolor{pink}{Wien}, höre von \textcolor{pink}{Berlin}, höre endlich den »\textcolor{green}{Kakadu}«,
                        \label{K_L00849_1v}\edtext{leſe wohl eine \textcolor{green}{venezianiſche Comödie}
                        vor}{\lemma{\textnormal{\emph{leſe … vor}}}\Cendnote{\textnormal{Am 30. 10. 1898
                        las er \emph{\textcolor{green}{Der Abenteurer und die Sängerin}}{ }\textcolor{blue}{Schnitzler} und \textcolor{blue}{Beer-Hofmann} vor.}}}\label{K_L00849_1h}, erzähle von \textcolor{blue}{\textsc{d’Annunzio}}, und ſage wie {\pb}alle
                    Herbſte aber noch mit viel tieferer Überzeugung als früher, daſs man ſich öfter
                    ſehen muſs.\pend
           \pstart
           Herzlich Ihr{\\[\baselineskip]}\spacefill\mbox{Hugo.}\pend
           \leftskip=0em{}\endnumbering\briefempfaengerindex{Schnitzler, Arthur@\textsc{Schnitzler, Arthur}!zzzHofmannsthal, Hugo von@\emph{von Hugo von Hofmannsthal}!1898-10-021@{2. 10. {[}1898{]}}|)be}\mylabel{h}  \input{latex-korrekturansicht-abspann}