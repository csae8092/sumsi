\input{latex-korrekturansicht-vorspann}
               \section[Hugo von Hofmannsthal an Arthur Schnitzler, 12. 10. {[}1898{]}]{ Hugo von Hofmannsthal an Arthur Schnitzler, 12. 10. {[}1898{]}}\nopagebreak\mylabel{v}\rehead{ }\normalsize\beginnumbering\briefempfaengerindex{Schnitzler, Arthur@\textsc{Schnitzler, Arthur}!zzzHofmannsthal, Hugo von@\emph{von Hugo von Hofmannsthal}!1898-10-121@{12. 10. {[}1898{]}}|(be} \toendnotes[C]{\smallbreak\pagebreak[2]} \Standort{CUL, Schnitzler, B 43.}
\physDesc{Brief, 1 Blatt, 3 Seiten
\newline{}Handschrift: schwarze Tinte, deutsche Kurrent
\newline{}Schnitzler: mit Bleistift die Jahreszahl ergänzt: »98« \newline{}Ordnung: mit Bleistift von unbekannter Hand nummeriert:
                                        »135« }\buchAbdrucke{\weitereDrucke{Hugo von Hofmannsthal, Arthur Schnitzler: \emph{Briefwechsel}. Hg. Therese Nickl und Heinrich Schnitzler. Frankfurt am Main: \emph{S. Fischer} 1964, S. 112.} }\toendnotes[C]{\smallbreak}\pstart
           {\pb}12. X.\hfill \textcolor{pink}{Gießhüblerstraße 2}\pend
           \pstart{}mein lieber Arthur\pend\pstart
           ich bin überaus froh, daſs es in \textcolor{pink}{Berlin}{ }ſo
                    abſolut gut gegangen iſt, denn ich habe für den zweiten und dritten \textcolor{green}{Act} große Angſt gehabt.\hspace*{1.5em}Mein \textcolor{pink}{venezianiſches} halb-ernſtes \textcolor{green}{Stück} iſt nahezu fertig. Ich bin nun noch für 5–6 Tage
                    hier, weil es ſo wunderſchön iſt, zwiſchen {\pb}den purpurrothen und gelben
                    Bäumen radzufahren. Es wäre ſo lieb von Ihnen wenn Sie einen der Wochentage in
                    der Früh herauskämen und bis zum Dunkelwerden hier blieben. Sie wiſſen daſs die
                        \textcolor{blue}{Schleſingers} darin keinen auf
                    ſie bezüglichen Beſuch {\pb}ſehen. Ich hätte eine ſehr große Freude darüber. Sie müſsten nur den Abend
                    vorher telegraphieren.\pend
           \pstart
           Von Herzen Ihr{\\[\baselineskip]}\spacefill\mbox{Hugo.}\pend
           \leftskip=0em{}\endnumbering\briefempfaengerindex{Schnitzler, Arthur@\textsc{Schnitzler, Arthur}!zzzHofmannsthal, Hugo von@\emph{von Hugo von Hofmannsthal}!1898-10-121@{12. 10. {[}1898{]}}|)be}\mylabel{h}  \input{latex-korrekturansicht-abspann}